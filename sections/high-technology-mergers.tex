\section{High Technology Mergers}

\subsection{FTC/DOJ, 2010 Horizontal Merger Guidelines}

\begin{enumerate}
    \item Merger analysis: concerned with the competitive effects of mergers. 
    Mergers should not be permitted to enhance market power.
    \item \textbf{Unilateral effects}.\footnote{See p. 20 ff.}
    \begin{enumerate}
        \item Elimination of competition between competitors.
        \item \textbf{Pricing of differentiated products}:\footnote{p. 20} if 
        the merging firms sell products that directly compete, the merger 
        could allow them to raise prices on both products.
        \item \textbf{Bargaining and auctions}:\footnote{p. 22} if two 
        sellers merger, buyers will no longer be able to play the sellers off 
        of each other.
        \item \textbf{Capacity and output for homogenous 
        products}:\footnote{p. 22} the merged firm may be able to suppress 
        output and raise the market price.
        \item \textbf{Innovation and product variety}: the merged firm may 
        lose incentives to innovate because it won't need to compete on 
        innovation.
    \end{enumerate}
    \item \textbf{Coordinated effects}.\footnote{p. 24 ff.}
    \begin{enumerate}
        \item Increasing the risk of coordinated anticompetitive behavior.
        \item First: is the market moderately or highly 
        concentrated?\footnote{p. 25.}
        \item Second: is the market vulnerable to coordinated 
        conduct?\footnote{p. 25 ff.}
        \item If yes to both, there is a strong risk of coordinated effects.
    \end{enumerate}
\end{enumerate}

\subsection{Horizontal Merger Analysis: \emph{U.S. v. Oracle Corp.}}

The DOJ tried to stop Oracle from acquiring PeopleSoft. It tried to show 
unilateral effects in differentiated products, but it failed because its 
market definition was too narrow.

\begin{enumerate}
    \item The DOJ and states sought to prevent Oracle from acquiring 
    PeopleSoft.
    \item Software at issue: ``enterprise application software'' (EAS). Three 
    kinds:
    \begin{enumerate}
        \item Mass market PC apps.
        \item Specially designed legacy software.
        \item \textbf{Enterprise resource planning (ERP)} software---at issue 
        here. Specifically, human relations management (HRM) and financial 
        management systems (FMS).
    \end{enumerate}
    \item Plaintiffs argued for the existence of ``high function HRM and FMS 
    software.''\footnote{p. 2.} They argued that this was a separate and 
    distinct market from other ERP software.\footnote{p. 4.}
    \item \textbf{Horizontal merger analysis}:
    \begin{enumerate}
        \label{oracle-unilateral-effects}
        \item \textbf{Unilateral effects}: four elements of a differentiated 
        products unilateral effects claim---(1) products must be 
        differentiated, (2) products must be close substitutes, (3) products 
        made by firms other than the merging firms must be sufficiently 
        different so that th merged firm can impose SSNIP, and (4) 
        repositioning by the non-merging firms must be unlikely.
    \end{enumerate}
    \item Held: the market definition was broader than the plaintiffs 
    alleged.\footnote{p. 12, 15, 17.} Other vendors (e.g., SAP) would help 
    maintain competition in the marketplace.
\end{enumerate}

\subsection{Google's Acquisition of Motorola Mobility}

\subsubsection{Backround}

\begin{enumerate}
    \item 2011: Google agreed to acquire Motorola Mobility, which included 
    around 17,000 patents, 6,800 patent applications, and \textbf{several 
    hundred standard essential patents}.
\end{enumerate}

\subsubsection{US Analysis}

\begin{enumerate}
    \item The acquisition is unlikely to substantially lessen competition.
    \item Google made ``ambiguous'' commitments to use the SEPs fairly, but 
    the FTC will ``monitor the use of SEPs'' and take enforcement actions if 
    appropriate.
\end{enumerate}

\subsubsection{EU Analysis}

\begin{enumerate}
    \item \textbf{Operating systems Relevant market}: operating systems for 
    ``feature phones'' and tablets.\footnote{p. 7.} Geographic market is 
    worldwide.\footnote{p. 8.}
    \item \textbf{Mobile devices relevant market}: smart mobile devices, and 
    maybe tablets.\footnote{p. 9.} Geographic market is worldwide.\footnote{p. 
    10.}
    \item Conclusion: no opposition.
\end{enumerate}
