\section{Horizontal Agreements}

\subsection{Hardcore Cartels and Efficient Collaborations}

\subsubsection{\emph{United States v. Andreas}}

% TODO 

\subsubsection{\emph{Broadcast Music, Inc. v. Columbia Broadcasting System, Inc.}}

% TODO 

\subsection{New Business Models}

\subsubsection{\emph{United States v. Adobe Systems, Inc.}}

% TODO 

\newpage % TODO remove
\subsubsection{\emph{United States v. Apple, Inc.}}

\begin{enumerate}
    \item The DOJ and 33 states claimed Apple and five book publishing 
    companies conspired to fix ebook prices in violation of \S\ 1 of the 
    Sherman Act.
    \item Facts:
    \begin{enumerate}
        \item The publishers were unhappy about Amazon's \$9.99 price for 
        ebooks.\footnote{p. 11.}
        \item Before April 2010, all publishers used the \textbf{wholesale  
        model}, in which the publisher sets a list price (i.e., a suggested 
        retail price) and then sells the books to a retailer at a wholesale 
        price (usually a percentage of the list price). The retailer then sells 
        the book at whatever price it chooses.\footnote{p. 14.} Amazon's \$9.99 
        was equal to or less than wholesale price of many of its ebooks, meaning 
        it was often selling ebooks as loss leaders.\footnote{pp. 15--17.}
        \item In 2009, the publishers started \textbf{windowing} (delaying ebook 
        publication after the release of the printed version). The publishers 
        synchronized the announcement of their windowing strategies.\footnote{p. 
        21.} But they believed windowing was not a long-term solution to 
        Amazon's \$9.99 price because it promoted piracy and penalized ebook 
        customers.
        \item Apple planned to announce the iPad on January 27, 2010. It wanted 
        to include an iBookstore. It assumed it would buy ebooks under the 
        wholesale model. Eddy Cue, SVP at Apple, began negotiating with the Big 
        Six.\footnote{p. 32.}
        \item Cue proposed an \textbf{agency model} in which the publishers 
        would set the price and Apple would earn a percentage of each sale. 
        Apple also required \textbf{price caps}.\footnote{p. 39.}
        \item Apple realized that it would be at a disadvantage if Amazon's 
        agreements with the publishers continued to use the wholesale model, 
        because Amazon could continue to undercut Apple's prices.  ``To ensure 
        that the iBookstore would be competitive at higher prices, Apple 
        concluded that it needed to eliminate all retail price competition. 
        Thus, the final component of its agency model required the Publishers to 
        move all of their e-tailers to agency.''\footnote{pp. 39--40.}
        \item Specifically, Cue proposed an agency model with a 30\% margin for 
        Apple, price caps, and a requirement that the Publishers switch all of 
        their other ebook agreements to the agency model.\footnote{p. 42.}
        \item Apple realized it could use a \textbf{most-favored nation (MFN) 
        clause} to force the publishers to adopt agency agreements with Amazon. 
        The MFN clause guaranteed that Apple's ebook retail prices would be the 
        lowest on the market. Amazon could continue to sell ebooks at \$9.99, 
        but Apple could match the price (and still earn 30\%). So, if the 
        publishers wanted to accomplish their goal of raising ebook prices 
        market-wide, they would have to adopt the agency model with Amazon to 
        force Amazon to raise its prices.\footnote{pp. 47--48.}
        \item The agency model would always be less profitable for publishers 
        than the wholesale model. (For instance, say Amazon and Apple both sell 
        the ebook at retail for \$12.99. Under the wholesale agreement with 
        Amazon in which the publisher's wholesale price to Amazon is \$13, the 
        publisher earns \$13 per book. But under the agency agreement with 
        Apple, the publisher would only receive 70\% of the revenue, or \$9.10.) 
        So the publishers would not have adopted the agency model unless it 
        suited their long-term interest of raising prices 
        market-wide.\footnote{pp. 53--54.}
        \item Apple secured contracts with four of the Big Six publishers, all 
        of which soon told Amazon that they wanted to switch to the agency 
        model.\footnote{p. 69, 73.} HarperCollins was the last holdout. Steve 
        Jobs convinced James Murdoch (of News Corp, which owned HC) that the 
        Apple would help the publishers form a united front against Amazon 
        (``so there is no leap of faith here''\footnote{p. 78.}). HC joined.
        \item As soon as the iPad launched, ``the coordinated pressure on Amazon 
        began at once.''\footnote{p. 83.}
        \item Jobs acknowledged to a reporter that the goal was to eliminate 
        competition from Amazon and raise ebook prices.\footnote{p. 86.}
        \item Soon, the ebooks saw ``sudden and uniform price increases'' across 
        the market.\footnote{p. 94.} ``The publisher defendants used the change 
        to an agency method for distributing their e-books as an opportunity to 
        raise the prices for their e-books across the board.''\footnote{p. 96.}
        \item The publishers lost sales as a result of the price increases, and 
        consumers suffered because they had to pay more for ebooks, buy cheaper 
        ebooks, or defer purchases entirely.\footnote{p. 98.}
    \end{enumerate}
    \item Holding:
    \begin{enumerate}
        \item The Sherman Act only outlaws \emph{unreasonable} restraints on 
        trade. Some offenses are per se anticompetitive, while others are 
        analyzed under the rule of reason. Courts have to weigh the competitive 
        effects of the agreement.\footnote{pp. 105--106.}
        \item Horizontal price fixing is illegal per se.\footnote{p. 107.}
        \item A price-fixing agreement requires a meeting of the 
        minds.\footnote{p. 109.}
        \item The court here found ``overwhelming evidence'' of a price-fixing 
        conspiracy.\footnote{p. 113.} Plaintiffs ``have proven a per se 
        violation of the Sherman Act.''\footnote{p. 120.}
    \end{enumerate}
    % TODO apple's arguments: p. 122 ff.
\end{enumerate}

