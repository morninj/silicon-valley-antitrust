\section{Introduction}

\subsection{Basics}

\begin{enumerate}
    \item Antitrust laws \textbf{enforce competition} to promote economic growth.
    \item Firms with more \textbf{market power} have more incentive and ability 
    to subvert competition. They are subject to certain \textbf{baseline 
    prohibitons}.
\end{enumerate}

\subsection{Offenses}

\begin{enumerate}
    \item Some offenses are \textbf{per se illegal}, regardless of market power. 
    For example:
    \begin{enumerate}
        \item Price fixing.
        \item Market allocation (territory, customer base, time).
        \item Group boycotts (i.e., refusals to deal).
    \end{enumerate}
    \item Baseline prohibitions apply to \textbf{net-anticompetitive 
    agreements}:
    \begin{enumerate}
        \item Tying and bundling.
        \item Exclusive dealing.
        \item Merges, acquisitions, joint ventures.
    \end{enumerate}
    \item Monopolists have special \textbf{unilateral conduct obligations}:
    \begin{enumerate}
        \item Refusals to deal with competitors.
        \item Hair-trigger standard for tying, exclusive dealing, and mergers.
    \end{enumerate}
\end{enumerate}

\subsection{Legal Provisions}

\subsubsection{Sherman Act \S\ 1}

\begin{enumerate}
    \item No agreements in restraint of trade.
    \item \textbf{Per se offenses}: independent of market power---price fixing, 
    market allocation, boycotts.
    \item \textbf{Rule of reason}: applies only to firms with mid-to-high market 
    power.
\end{enumerate}

\subsubsection{Sherman Act \S\ 2}

\begin{enumerate}
    \item No abuse of a monopoly position. Requires \textbf{monopoly 
    power}\footnote{A flexible standard. Around 50\% market share in the US, 
    40\% in the EU.} and \textbf{unilateral exclusionary conduct}.
\end{enumerate}

\subsubsection{Clayton Act \S\ 7}

\begin{enumerate}
    \item No acquisition of shares or assets that would likely result in a 
    ``substantial lessening of competition.''
\end{enumerate}

\subsection{Examples of Anticompetitive Conduct}

\subsubsection{Coordination}

\begin{enumerate}
    \item If two firms are bidding to sell to a customer, they agree not to 
    compete.
\end{enumerate}

\subsubsection{Exclusion}

\begin{enumerate}
    \item If two firms buy from suppliers, the suppliers agree not to sell to 
    one of the firms.
\end{enumerate}

\subsubsection{Leveraging}

\begin{enumerate}
    \item A seller uses one of its products to increase sales for another 
    product---e.g., bundling a media player with an operating system.
\end{enumerate}
