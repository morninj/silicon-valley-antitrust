\section{U.S. v. Microsoft}

% TODO market definition -- see baker, 'market definition'; and see sager

\subsection{Key Points}

\begin{enumerate}
    \item Centers on Microsoft's various actions to defend its OS monopoly 
    against the emergence of interoperability layers.
    \item Indirect network effects (``application barrier to entry'') helped 
    preserve Microsoft's monopoly by creating a barrier to entry (i.e., 
    developers wanted to code for the platform with the most users). Java and 
    Netscape (?) provided interoperability layers that threatened its 
    dominance.
    \begin{enumerate}
        \item \emph{Direct network effects}: the more users, the more valuable 
        the platform is for the users. E.g., Facebook.
        \item \emph{Indirect network effects}: the more users, the more 
        valuable the platform is for another constituency, like developers. 
        E.g., Facebook again.
    \end{enumerate}
    \item DOJ won, but Microsoft avoided harsh remedies.
    \item Interoperability layers can be \textbf{nascent threats}. This case 
    put nascent threats in a special protected antitrust class.
\end{enumerate}

\subsection{Analyzing a \S\ Monopolization Problem}

\begin{enumerate}
    \item Monopoly power.
    \begin{enumerate}
        \item Direct proof, and/or
        \item Indirect proof:
        \begin{enumerate}
            \item Market definition (hypothetical monopolist + small but 
            significant nontransitory increase in price [SSNIP])---i.e., an 
            actor with the ability to profitably raise prices. The market 
            includes reasonable substitutes.
            \begin{enumerate}
                \item Avoid the \emph{Cellophane} fallacy: if a monopolist has 
                had a monopoly for a long time, then it will already be 
                charging a monopoly price, so this test fails. There's no good 
                way to deal with this.
            \end{enumerate}
            \item Significant market share (somewhere around 60\% or more).
            \item Barriers to entry.
        \end{enumerate}
    \end{enumerate}
    \item Exclusionary conduct.
    \begin{enumerate}
        \item Harms rivals = anticompetitive effect (AE).
        \item Does not benefit consumers = procompetitive efficiency. (TODO: 
        verify. See HK slides 04-02, slide 10, at 
        http://www.hannokaiser.com/lawschool/2013-fall-svat/slides/2013-svatf-04-02.pdf)
    \end{enumerate}
    \item Causation.
    \begin{enumerate}
        \item The exclusionary conduct creates or reinforces the monopoly 
        power.
    \end{enumerate}
\end{enumerate}

\subsection{Issues}

\subsubsection{Network Effects}

\begin{enumerate}
    \item Network effects can be a barrier to entry.
    \item Direct vs. indirect---see above.
\end{enumerate}

\subsubsection{Interoperability}

\begin{enumerate}
    \item Java and Netscape threatened Microsoft's OS monopoly. Microsoft 
    wanted to stunt their growth.
\end{enumerate}

\subsubsection{Anticompetitive Conduct}

\begin{enumerate}
    \item ``~.~.~.~it must harm the competitive \emph{process} and thereby 
    harm consumers~.~.~.~''\footnote{p. 58.}
    \item ``~.~.~.~the means of illicit exclusion, like the means of 
    legitimate competition, are myriad~.~.~.~''\footnote{p. 58.}
\end{enumerate}

\subsubsection{OEM Licenses}

\begin{enumerate}
    \item Microsoft entered licensing agreements with OEM manufacturers to 
    prevent them from loading Windows with ``drastic alterations.''
    \item Microsoft said it was just exercising its IP rights, but the court 
    held that this was like arguing that using one's personal property, like a 
    baseball bat, cannot give rise to tort liability.
\end{enumerate}

\subsubsection{Integration of IE and Windows}

\begin{enumerate}
    \item See p. 64 ff.
\end{enumerate}

\subsubsection{Java}

\begin{enumerate}
    \item Microsoft developed its own JVM.
    \item Its strategy: embrace, extend, extinguish.
\end{enumerate}

\subsubsection{Causation}

\begin{enumerate}
    \item The court employed a loose, almost homeopathic causation standard. 
    It held that if excluding nascent threats is the type of conduct that is 
    reasonably capable of contributing to the continuation of monopoly power, 
    it can count as causation.
    \item Significantly different than the usual but-for causation standard in 
    criminal law (and this is a criminal case). The same court would later 
    strike down this standard in another case.
    \item ``Nascent threat'' jurisprudence: sometimes nascent threats are 
    entitled to more protection than small but established competitors.
\end{enumerate}
