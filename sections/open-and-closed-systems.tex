\section{Open and Closed Systems}

\subsection{Aftermarkets}

\subsubsection{Bait-and-Switch: \emph{Eastman Kodak Co. v. Image Technical 
Services, Inc.}}

Generally, competition in the primary market (e.g., equipment) protects 
customers in the aftermarket (e.g., service for the equipment), even when a 
firm has monopoly power in the aftermarket (e.g., Kodak here). So, there is 
\textbf{generally no \S\ 2 claim in the aftermarket}.

But sometimes the equipment market fails to discipline conduct in the 
aftermarket (e.g., the equipment market did not punish Kodak for its 
exclusionary conduct in the aftermarket). This can happen when (1) customers 
are \textbf{locked in} to a product in the primary market, or (2) customers 
have \textbf{insufficient knowledge} at the time of the purchase (e.g., when 
Kodak enacted a new policy that closed a formerly open system).

The ``insufficient knowledge'' exception fails if (1) there was an agreement 
spelling out the aftermarket restrictions (e.g., franchising---\emph{Queen 
City Pizza}) or (2) the aftermarket products or services were purchased at the 
time of the primary equipment purchase.

\begin{enumerate}
    \item Facts:
    \begin{enumerate}
        \item Kodak sells photocopiers and micrographic equipment. It also 
        sells parts and services.
        \item Kodak parts are only compatible with Kodak machines.
        \item Kodak has many competitors in the copier market. Its market 
        share in the copier market is less than 30\%.
        \item Independent service operators (ISOs) sell services for Kodak 
        machines. They buy parts from Kodak and from OEMs.
        \item 1985/1986: Kodak tried to freeze out the ISOs by implementing a 
        policy change to move from an \textbf{open system to a closed system}:
        \begin{enumerate}
            \item It implemented a \textbf{new policy} of selling replacement 
            parts only to buyers who use Kodak to service or repair their 
            machines.  ISOs could still get parts from OEMs; but then Kodak 
            adjusted its OEM contracts to prevent OEMs from selling parts to 
            ISOs (which is not an antitrust problem in itself---exclusive 
            dealing is allowable).
            \item Kodak customers who bought Kodak parts were then prevented 
            contractually from buying services from ISOs; so, they were 
            required to buy service from Kodak.
            \item (Kodak's change created many new motivated plaintiffs.)
        \end{enumerate}
        \item Here, 18 ISOs challenged Kodak's new policy.
    \end{enumerate}
    \item Claims:
    \begin{enumerate}
        \item \textbf{Tying} (first claim):
        \begin{enumerate}
            \item Four elements:\footnote{p. 462.}
            \begin{enumerate}
                \item \textbf{Two products}.
                \begin{itemize}
                    \item First product: Kodak \emph{parts} (not copiers).
                    \item Second product: Kodak services.
                    \item (They're separate products because they pass the 
                    ``separate demand'' test---i.e., there are buyers who buy 
                    one and not the other.)
                \end{itemize}
                \item \textbf{Tie}.
                \begin{itemize}
                    \item The two products here---parts and services---are 
                    \textbf{tied by contract}.
                    \item (Other types of ties are technological (Windows/IE) 
                    and economic (e.g., a higher price for the unbundled 
                    alternative; it would be economically irrational to buy 
                    the more expensive unbundled products).)
                \end{itemize}
                \item \textbf{Market power} in the \emph{tying} product market.
                \begin{itemize}
                    \item The tying product is the must-have product---e.g., 
                    Microsoft Windows. You're then tempted to tie another 
                    product to it.  
                    \item Here, the Kodak parts are the tying products.
                \end{itemize}
                \item \textbf{Some effect} in the \emph{tied} product market.
                \begin{itemize}
                    \item Here, the tied product was services for Kodak 
                    copiers.
                \end{itemize}
            \end{enumerate}
        \end{enumerate}
        \item \textbf{Monopolization of the parts and services markets} 
        (second claim):
        \begin{enumerate}
            \item \textbf{Monopoly power}: yes (100\% market share in parts, 
            80-90\% in services).
            \item \textbf{Exclusionary conduct}: yes---it forced out the ISOs.
            \item \textbf{Plausible business justifications?} No.
            \begin{enumerate}
                \item Quality? No, customers preferred ISO.
                \item Inventory management? No.
                \item Trying to keep ISOs from free riding? No.
            \end{enumerate}
        \end{enumerate}
    \end{enumerate}
    \item Kodak has a 100\% market share in the replacement parts market.
    \item \emph{Effect on the tied product market}: prices went up; ISOs 
    were driven out; customers were \textbf{forced} to by inferior Kodak 
    services.
    \item But (this is the innovative part of the \emph{Kodak} opinion):
    \begin{enumerate}
        \item Kodak argued, as a matter of law, that because it does not 
        have market power in the \emph{equipment} market, it cannot have 
        market power in the \emph{parts} market, because buyers would stop 
        buying Kodak equipment.\footnote{p. 466.} 
        \item Held: buyers may not know about the service price increases 
        (e.g., if the buyer of a copier is not the buyer of services).
        \item Customers may not know the lifecycle costs of a copier 
        (``total cost of ownership''---TCO). \textbf{Customers couldn't do 
        a TCO calculation because the Kodak service prices jumped 
        \emph{after} they bought the equipment.}
        \begin{enumerate}
            \item The bait-and-switch---the change in policy---is the core 
            of the case.
        \end{enumerate}
        \item Competition in the primary market (copiers) may not protect 
        competition in the aftermarket (services) if there are information 
        problems (here, the information problem was that customers could not 
        calculate the cost of ownership). Moreover, the equipment here was 
        expensive and locked customers in to ownership of Kodak equipment.
        \begin{enumerate}
            \item So, perhaps, Kodak had monopoly power under \S\ 2 in 
            the parts market.
        \end{enumerate}
    \end{enumerate}
    \item Usually rule: equipment market competition protects aftermarket 
    customers. (I.e., customers usually punish sellers that harm customers in 
    the aftermarket.) But, two exceptions: 
    \begin{enumerate}
        \item \textbf{Exception one: insufficient information.} Insufficient 
        information at the time of the equipment purchase can make it hard to 
        customers to punish sellers, so aftermarket behavior can violate \S\ 
        2. This exception would not apply if it's in the context of a ``clear 
        contract.'' E.g., franchising (\emph{Queen City Pizza}\footnote{Cited 
        in \emph{Newcal}.}), which includes contracts dictating where the 
        franchisee has to buy ingredients---because there \emph{was} 
        sufficient information at the time of the purchase. What if the 
        equipment and the aftermarket purchase happen at the same time (e.g., 
        prepay for a warranty and parts when you buy a copier)? The buyer also 
        has complete information in this scenario.
        \item \textbf{Exception two: lock in.} If you're locked into ownership 
        of a product, \S\ 2 may apply.
    \end{enumerate}
    \item Justice Scalia, dissenting:
    \begin{enumerate}
        \item Kodak only has power over buyers because the buyers bought from 
        Kodak in the first place. Isn't this just an ordinary contract? That's 
        significantly different than companies that monopolize an entire 
        market.---In other words: this is \emph{contract} power, not 
        \emph{market} power.
    \end{enumerate}
\end{enumerate}

\subsubsection{\emph{Newcal Industries v. IKON Office Solutions}}

Aftermarket doctrine in the 9th Circuit: if there was \textbf{insufficient 
disclosure} of what would happen in the aftermarket at the time of purchase, 
you \emph{may} have a \S\ 2 violation.

\begin{enumerate}
    \item Facts:
    \begin{enumerate}
        \item IKON sells (leases) copiers and services. They need parts. 
        Newcal is an ISO, providing services.
        \item Service contracts are entered into when the lease agreements 
        begin. (See \emph{Kodak} on contractual agreements.)
        \item Leasing agreements eventually expired. Then, there's a new round 
        of competition for leasing and service agreements. IKON asked 
        customers to extend leasing terms, with allegedly deceptive clauses 
        that also extended servicing terms.
        \item \emph{Tying} product: leasing agreement. \emph{Tyed} product: 
        service agreement.
    \end{enumerate}
    \item Held: not analogous to \emph{Queen City Pizza} because at the time 
    of the renewal agreements, the lessee was not properly informed that they 
    would be extending their service agreement. This is more like \emph{Kodak} 
    because the customer was not properly informed.
    \item Ninth Circuit test for insufficient information at the time of 
    equipment purchase:
    \begin{enumerate}
        \item If there was \textbf{insufficient disclosure} of what would 
        happen in the aftermarket at the time of purchase, you \emph{may} have 
        a \S\ 2 violation.
    \end{enumerate}
\end{enumerate}

\subsubsection{Hanno Kaiser, ``Are `Closed Systems' an Antitrust Problem?''}

\begin{enumerate}
    \item Should we treat ``closed systems'' as inherently anticompetitive?
    \item It's not easy to define what's open and what's closed. A fully open 
    system is not a system at all, and every ``closed'' system must connect to 
    something external, like the power grid. Also, computer systems have 
    multiple layers which may be more or less open.\footnote{93--94.}
    \item Vertical restraints are not in an antitrust suspect class. Courts 
    treat them with ``great leniency.''\footnote{95.}
    \item Computer platforms have three components: sponsors, contributors, 
    and users. There are two relevant realms of competition: \emph{among} 
    ecosystems and \emph{within} ecosystems.\footnote{96.}
    \item Intra-platform rules can limit negative externalities---e.g., 
    mandating security standards for third-party developers reduces security 
    risks for the platform as a whole.\footnote{98.} These restraints can make 
    a platform more competitive against other platforms.\footnote{99.}
    \item Common platform rules govern (1) quality and content, (2) security, 
    (3) privacy, and (4) technology.\footnote{99--102.}
    \item Are these intra-platform restraints (by sponsors on contributors) 
    exclusionary?\footnote{102.} No. They can lead to anticompetitive 
    exclusion, but only if the sponsor has significant market power.
    \item ``~.~.~.~it is by no means clear that open systems are preferable to 
    closed systems in competitive markets.''\footnote{103.}
\end{enumerate}

\subsubsection{Hackintosh: \emph{Apple Inc. v. Psystar Corp.}}

Psystar failed to plausibly define the relevant markets. First, Psystar did 
not plausibly allege the existence of a single-brand Mac OS market. Second, 
``hardware systems that utilize Mac OS'' is not a relevant market because 
Apple's contractual restraints are consistent with fair competition (analogous 
with \emph{Queen City Pizza}, and distinct from \emph{Kodak}).

\begin{enumerate}
    \item Psystar manufactured computers capable of running a range of 
    operating systems, including Mac OS. No companies other than Apple and 
    Psystar sold computers compatible with Mac OS. Psystar argued that Apple's 
    EULA and anti-competitive market practices meant that customers who wanted 
    to use Mac OS had no alternative to Apple hardware.\footnote{1193.}
    \item Psystar claimed (1) unlawful tying in violation of \S\ 1 of the 
    Sherman Act and (2) monopoly maintenance in violation of \S\ 
    2.\footnote{1195.} Apple moved to dismiss.
    \item \emph{Twombly} pleading standard: stating a claim requires 
    \emph{plausible grounds}.\footnote{1195.}
    \item Apple argued (1) that the relevant markets were neither legally nor 
    factually plausible and (2) Apple is not required to help its competitors 
    by entering unwanted license agreements with them.\footnote{1195.}
    \item Market definition of the ``Mac OS market'':
    \begin{enumerate}
        \item Can Psystar rely on a relevant market comprised of a single 
        brand of a product?
        \item Whether products are in the same market depends on whether 
        consumers view them as \textbf{reasonable substitutes} and whether 
        they \textbf{would switch among them} in response to price changes. 
        \footnote{1196.}
        \item \emph{Newcal} dealt with a \emph{single-brand aftermarket}. 
        Here, the dispute centers on a \emph{primary market}.\footnote{1197.}
        \item \emph{Kodak} dealt with a \emph{derivative aftermarket} (i.e., 
        companies that service Kodak machines). But again, here the dispute 
        centers on an independent, primary market.
        \item Under ``rare and unforeseen circumstances,'' there may be a 
        market consisting of only one brand, but Psystar has not met the 
        \emph{Twombly} standard in pleading a plausible claim of such a 
        market.\footnote{1198.}
        \item Psystar argues that a SSNIP would not change demand for Mac 
        OS---but that does not prove a lack of 
        competition.\footnote{1198--1200.}
        \item \textbf{``~.~.~.~the counterclaim does not plausibly allege that 
        Mac OS is an independent market.''}\footnote{1200.}
    \end{enumerate}
    \item Market definition of the ``Mac OS--capable computers market'':
    \begin{enumerate}
        \item Is ``hardware systems that utilize Mac OS'' a relevant 
        market?\footnote{1200.}
        \item No. Instead, the hardware market at issue is analogous to 
        \emph{Queen City Pizza}, in which contractual restraints are 
        consistent with fair competition.\footnote{1202.}
    \end{enumerate}
\end{enumerate}

\subsubsection{\emph{Datel Holdings Ltd. v. Microsoft Corp.}}

\begin{enumerate}
    \item Datel sold aftermarket accessories for the Xbox. Microsoft released 
    a software update that rendered Datel's accessories useless. Microsoft 
    moved to dismiss.
    \item Alleged relevant markets: (1) aftermarket for Xbox 360 accessories 
    and add-ons (``aftermarket'') and (2) multiplayer online dedicated gaming 
    systems market (``online market'').
    \item Legal standard: \emph{Twombly} plausibility.\footnote{982.}
    \item First claim---violation of the Sherman Act in the aftermarket:
    \begin{enumerate}
        \item Is there a single-brand aftermarket? Under \emph{Newcal} and 
        \emph{Kodak}, only if consumers lacked sufficient information to bind 
        themselves knowingly and voluntarily. Otherwise, it's a contractual 
        agreement, consistent with fair competition (\emph{Queen City Pizza}). 
        \item Held: there were ambiguities in the contractual language. So, 
        the motion to dismiss on that claim was denied.\footnote{990.}
    \end{enumerate}
    \item Second claim---violation of the Sherman Act in the online market:
    \begin{enumerate}
        \item Plaintiffs lack antitrust injury and therefore lacks standing. 
        Motion to dismiss was granted.\footnote{994.}
    \end{enumerate}
    \item Third claim---tying:
    \begin{enumerate}
        \item Datel argued that Microsoft illegally tied the Xbox to 
        accessories and add-ons. Microsoft argued that Datel failed to plead a 
        viable market and aftermarket---it should include other platforms, 
        like the Wii and personal computers, and not just the Playstation.
        \item Held: the tying claim was adequately pled. Motion to dismiss 
        denied.\footnote{999.}
    \end{enumerate}
\end{enumerate}
