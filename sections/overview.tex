\section{Overview}

\subsection{Introduction} 

\begin{enumerate}
    \item \textbf{Basics}: output maximization, market power.
    \item \textbf{Increasing profits without maximizing output}: collusion, 
    exclusion, leveraging (examples: see p.~\pageref{sub:conduct-examples}).
    \item \textbf{Per se offenses}: price fixing, market allocation, boycotts 
    (and technically tying).
    \item \textbf{Rule of reason offenses}: baseline prohibitions against 
    net-anticompetitive activities (tying, bundling, exclusive dealing, 
    mergers, acquisitions, joint ventures).
    \item \textbf{Provisions}: Sherman Act \S\S\ 1 and 2, Clayton Act \S\ 7.
\end{enumerate}

\subsection{Horizontal Agreements}

\begin{enumerate}
    \item ``Unreasonable'' means AE > PE.
    \item \textbf{Per se illegal}: no need to show AE > PE. Price fixing, market 
    division, group boycotts. Classic example: \emph{US v. Andreas}.
    Other cases apply the \textbf{rule of reason} 
    analysis, which asks only if AE > PE.
    \item Courts will only apply the per se treatment if there has been 
    \textbf{significant judicial experience} in the area. \emph{Broadcast v. 
    BMI}.
    \item \textbf{Ancillary restraints analysis}: if the restrictive agreement 
    is \textbf{ancillary} to an underlying transaction, the court will analyze 
    it under the rule of reason.
    \item Are employee no-cold-all agreements per se unlawful? \emph{US v. 
    Adobe}.
    \item Apple's role in facilitating horizontal ebook deals with publishers 
    were per se violations of \S\ 1. \emph{US v. Apple}.
\end{enumerate}

\subsection{Microsoft}

\begin{enumerate}
    \item Centers on Microsoft's various actions to defend its OS monopoly 
    against the emergence of \textbf{interoperability layers}, which can be a 
    threat to two-sided market platform power.
    \item Specific issues---see p.~\pageref{sub:msft-issues}.
    \item \textbf{Indirect network effects} (``application barrier to entry'') 
    helped preserve Microsoft's monopoly by creating a barrier to entry (i.e., 
    developers wanted to code for the platform with the most users). Java and 
    Netscape (?) provided interoperability layers that threatened its 
    dominance.
    \begin{enumerate}
        \item \emph{Direct network effects}: the more users, the more valuable 
        the platform is \emph{for the users}. E.g., Facebook.
        \item \emph{Indirect network effects}: the more users, the more 
        valuable the platform is \emph{for another constituency}, like 
        developers. E.g., Facebook again.
    \end{enumerate}
    \item Interoperability layers can be \textbf{nascent threats}. This case 
    put nascent threats in a special protected antitrust class.
    \item DOJ won, but Microsoft avoided harsh remedies.
    \item Problem: \textbf{timely remedies}.
\end{enumerate}

\subsection{Predatory Innovation}

\begin{enumerate}
    \item Some innovation is exclusionary rather than innovative.m When should 
    courts police \textbf{product design decisions}, if at all? Should courts 
    prevent companies from \textbf{transferring monopoly power from one 
    product generation to the next}?
    \item \textbf{Cheap exclusion}: companies can make small changes to code 
    to exclude competitors. Is this different?
    \item Predatory redesigns can be especially harmful in network markets 
    (e.g., software and hardware) because of \textbf{lock-in}, \textbf{network 
    effects}, and \textbf{intellectual property rights}.
    \item \emph{Microsoft} test:
    \begin{enumerate}
        \item Plaintiff has the burden of establishing a prima facie case of 
        predation.
        \item Defendant can rebut the prima facie case.
        \item Plaintiff must show that exclusionary effects outweigh the 
        defendant's justifications.
    \end{enumerate}
    \item \emph{Allied Ortho v. Tyco} test (Ninth Circuit):
    \begin{enumerate}
        \item The Ninth Circuit held that one iota of benefit to the consumer 
        will justify the innovation, no matter how small it may be, no matter 
        whether it is predatory, and no matter its effect on the market.
        \item Only applies in the Ninth Circuit.
    \end{enumerate}
    \item \emph{FTC v. Intel}:
    \begin{enumerate}
        \item Claims: see p.~\pageref{sub:intel}.
        \item Consent decree: Intel cannot make changes to products covered by 
        the decree if the change (1) would degrade the performance of a 
        competitor's product and (2) does not provide an ``actual'' benefit to 
        Intel's product. The FTC specifically rejected the \emph{Allied Ortho} 
        approach, writing that in a legal challenge it would be appropriate to 
        balance the AEs and PEs.
    \end{enumerate}
    
\end{enumerate}

\subsection{Open and Closed Systems}

\begin{enumerate}
    \item \textbf{\emph{Kodak}}:
    \begin{enumerate}
        \item \textbf{\emph{Kodak} takeaway}: lack of market power in the 
        primary market does not preclude liability for exclusionary conduct in 
        aftermarkets.
        \item Generally, competition in the primary market (e.g., equipment) 
        protects customers in the aftermarket (e.g., service for the 
        equipment), even when a firm has monopoly power in the aftermarket 
        (e.g., Kodak here).  So, there is \textbf{generally no \S\ 2 claim in 
        the aftermarket}.  \emph{Kodak}.
        \item But sometimes the equipment market fails to discipline conduct 
        in the aftermarket (e.g., the equipment market did not punish Kodak 
        for its exclusionary conduct in the aftermarket). This can happen when 
        (1) customers are \textbf{locked in} to a product in the primary 
        market, or (2) customers have \textbf{insufficient knowledge} at the 
        time of the purchase (e.g., when Kodak enacted a new policy that 
        closed a formerly open system). \emph{Kodak}.
        \item The ``insufficient knowledge'' exception fails if (1) there was 
        an agreement spelling out the aftermarket restrictions (e.g., 
        franchising---\emph{Queen City Pizza}) or (2) the aftermarket products 
        or services were purchased at the time of the primary equipment 
        purchase.  \emph{Kodak}.
    \end{enumerate}
    \item Aftermarket doctrine in the Ninth Circuit: if there was 
    \textbf{insufficient disclosure} of what would happen in the aftermarket 
    at the time of purchase, you \emph{may} have a \S\ 2 violation.
    \item We should not treat closed systems as inherently anticompetitive. 
    See Kaiser below, p.~\pageref{sub:kaiser-closed-systems}.
    \item \textbf{Contractual restraints} are consistent with fair 
    competition. \emph{Apple v. Psystar} (applying \emph{Queen City Pizza}, 
    the franchising case).
    \item \emph{Datel v. Microsoft}: Datel overcame summary judgment by 
    showing (1) \textbf{contractual ambiguity}, raising the possibility of 
    insufficient information about the aftermarket, and (2) the tying claim 
    \textbf{adequately pled the relevant markets}.
\end{enumerate}

\subsection{Refusal to Deal}

\begin{enumerate}
    \item Monopolists generally are not required to deal with anyone, but 
    there are exceptions---e.g., common carrier regulations, 
    \textbf{discontinuation of a practice} (for companies with significant 
    market power---over 60\%---\emph{Aspen}), and the essential facilities 
    doctrine (which is rarely used).
    \item After \emph{Trinko}, it's unclear whether the plaintiff \emph{must} 
    show that the monopolist acted irrationally, or whether irrational conduct 
    \emph{may} support a finding of exclusionary conduct.
    \item De novo refusals to deal are fine. Obligations only arise from 
    \textbf{preexisting profitable courses of dealing}.
    \item (The \emph{essential facilities doctrine} lives on in some circuits. 
    But it has no practical significance in the U.S.)
    \item You generally have to show \textbf{profit sacrifice} to prove 
    exclusionary conduct.
\end{enumerate}

\subsection{Smartphone Wars} % TODO 

\begin{enumerate}
    \item 
\end{enumerate}

\subsection{Google Search Investigations} % TODO 

\begin{enumerate}
    \item 
\end{enumerate}

\subsection{High Technology Mergers} % TODO 

\begin{enumerate}
    \item 
\end{enumerate}
