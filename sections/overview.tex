\section{Overview}

\subsection{Introduction} 

\begin{enumerate}
    \item \textbf{Basics}: output maximization, market power.
    \item \textbf{Increasing profits without maximizing output}: collusion, 
    exclusion, leveraging (examples: see p.~\pageref{sub:conduct-examples}).
    \item \textbf{Per se offenses}: price fixing, market allocation, boycotts 
    (and technically tying).
    \item \textbf{Rule of reason offenses}: baseline prohibitions against 
    net-anticompetitive activities (tying, bundling, exclusive dealing, 
    mergers, acquisitions, joint ventures).
    \item \textbf{Provisions}: Sherman Act \S\S\ 1 and 2, Clayton Act \S\ 7.
\end{enumerate}

\subsection{Horizontal Agreements}

\begin{enumerate}
    \item ``Unreasonable'' means AE > PE.
    \item \textbf{Per se illegal}: no need to show AE > PE. Price fixing, market 
    division, group boycotts. Classic example: \emph{US v. Andreas}.
    Other cases apply the \textbf{rule of reason} 
    analysis, which asks only if AE > PE.
    \item Courts will only apply the per se treatment if there has been 
    \textbf{significant judicial experience} in the area. \emph{Broadcast v. 
    BMI}.
    \item \textbf{Ancillary restraints analysis}: if the restrictive agreement 
    is \textbf{ancillary} to an underlying transaction, the court will analyze 
    it under the rule of reason.
    \item Are employee no-cold-all agreements per se unlawful? \emph{US v. 
    Adobe}.
    \item Apple's role in facilitating horizontal ebook deals with publishers 
    were per se violations of \S\ 1. \emph{US v. Apple}.
\end{enumerate}

\subsection{Microsoft}

\begin{enumerate}
    \item Centers on Microsoft's various actions to defend its OS monopoly 
    against the emergence of \textbf{interoperability layers}, which can be a 
    threat to two-sided market platform power.
    \item Specific issues---see p.~\pageref{sub:msft-issues}.
    \item \textbf{Indirect network effects} (``application barrier to entry'') 
    helped preserve Microsoft's monopoly by creating a barrier to entry (i.e., 
    developers wanted to code for the platform with the most users). Java and 
    Netscape (?) provided interoperability layers that threatened its 
    dominance.
    \begin{enumerate}
        \item \emph{Direct network effects}: the more users, the more valuable 
        the platform is \emph{for the users}. E.g., Facebook.
        \item \emph{Indirect network effects}: the more users, the more 
        valuable the platform is \emph{for another constituency}, like 
        developers. E.g., Facebook again.
    \end{enumerate}
    \item Interoperability layers can be \textbf{nascent threats}. This case 
    put nascent threats in a special protected antitrust class.
    \item DOJ won, but Microsoft avoided harsh remedies.
    \item Problem: \textbf{timely remedies}.
\end{enumerate}

\subsection{Predatory Innovation}

\begin{enumerate}
    \item Some innovation is exclusionary rather than innovative.m When should 
    courts police \textbf{product design decisions}, if at all? Should courts 
    prevent companies from \textbf{transferring monopoly power from one 
    product generation to the next}?
    \item \textbf{Cheap exclusion}: companies can make small changes to code 
    to exclude competitors. Is this different?
    \item Predatory redesigns can be especially harmful in network markets 
    (e.g., software and hardware) because of \textbf{lock-in}, \textbf{network 
    effects}, and \textbf{intellectual property rights}.
    \item \emph{Microsoft} test:
    \begin{enumerate}
        \item Plaintiff has the burden of establishing a prima facie case of 
        predation.
        \item Defendant can rebut the prima facie case.
        \item Plaintiff must show that exclusionary effects outweigh the 
        defendant's justifications.
    \end{enumerate}
    \item \emph{Allied Ortho v. Tyco} test (Ninth Circuit):
    \begin{enumerate}
        \item The Ninth Circuit held that one iota of benefit to the consumer 
        will justify the innovation, no matter how small it may be, no matter 
        whether it is predatory, and no matter its effect on the market.
        \item Only applies in the Ninth Circuit.
    \end{enumerate}
    \item \emph{FTC v. Intel}:
    \begin{enumerate}
        \item Claims: see p.~\pageref{sub:intel}.
        \item Consent decree: Intel cannot make changes to products covered by 
        the decree if the change (1) would degrade the performance of a 
        competitor's product and (2) does not provide an ``actual'' benefit to 
        Intel's product. The FTC specifically rejected the \emph{Allied Ortho} 
        approach, writing that in a legal challenge it would be appropriate to 
        balance the AEs and PEs.
    \end{enumerate}
    
\end{enumerate}

\subsection{Open and Closed Systems} % TODO 

\begin{enumerate}
    \item 
\end{enumerate}

\subsection{Refusal to Deal} % TODO 

\begin{enumerate}
    \item 
\end{enumerate}

\subsection{Smartphone Wars} % TODO 

\begin{enumerate}
    \item 
\end{enumerate}

\subsection{Google Search Investigations} % TODO 

\begin{enumerate}
    \item 
\end{enumerate}

\subsection{High Technology Mergers} % TODO 

\begin{enumerate}
    \item 
\end{enumerate}
