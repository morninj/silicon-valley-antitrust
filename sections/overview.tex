\section{Overview}

\subsection{Introduction} 

\begin{enumerate}
    \item \textbf{Basics}: output maximization, market power.
    \item \textbf{Increasing profits without maximizing output}: collusion, 
    exclusion, leveraging (examples: see p.~\pageref{sub:conduct-examples}).
    \item \textbf{Per se offenses}: price fixing, market allocation, boycotts 
    (and technically tying).
    \item \textbf{Rule of reason offenses}: baseline prohibitions against 
    net-anticompetitive activities (tying, bundling, exclusive dealing, 
    mergers, acquisitions, joint ventures).
    \item \textbf{Provisions}: Sherman Act \S\S\ 1 and 2, Clayton Act \S\ 7.
\end{enumerate}

\subsection{Horizontal Agreements}

\begin{enumerate}
    \item ``Unreasonable'' means AE > PE.
    \item \textbf{Per se illegal}: no need to show AE > PE. Price fixing, market 
    division, group boycotts. Classic example: \emph{US v. Andreas}.
    Other cases apply the \textbf{rule of reason} 
    analysis, which asks only if AE > PE.
    \item Courts will only apply the per se treatment if there has been 
    \textbf{significant judicial experience} in the area. \emph{Broadcast v. 
    BMI}.
    \item \textbf{Ancillary restraints analysis}: if the restrictive agreement 
    is \textbf{ancillary} to an underlying transaction, the court will analyze 
    it under the rule of reason.
\end{enumerate}

\subsection{Microsoft} % TODO 

\begin{enumerate}
    \item 
\end{enumerate}

\subsection{Predatory Innovation} % TODO 

\begin{enumerate}
    \item 
\end{enumerate}

\subsection{Open and Closed Systems} % TODO 

\begin{enumerate}
    \item 
\end{enumerate}

\subsection{Refusal to Deal} % TODO 

\begin{enumerate}
    \item 
\end{enumerate}

\subsection{Smartphone Wars} % TODO 

\begin{enumerate}
    \item 
\end{enumerate}

\subsection{Google Search Investigations} % TODO 

\begin{enumerate}
    \item 
\end{enumerate}

\subsection{High Technology Mergers} % TODO 

\begin{enumerate}
    \item 
\end{enumerate}
