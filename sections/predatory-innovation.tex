\section{Predatory Innovation}

\subsection{\emph{Allied Orthopedic v. Tyco Healthcare}}

\begin{enumerate}
    \item Facts:\footnote{994--95.}
    \begin{enumerate}
        \item Tyco had a significant share of the market for pulse oximetry 
        monitors and sensors. Its ``R-Cal'' patent prevented competitors from 
        selling sensors compatible with its monitors.
        \item The patent was set to expire in November 2003. Tyco developed 
        ``OxiMax,'' a new patented sensor design that included a chip in the 
        sensor itself. It also developed new monitors to complement the 
        chipped sensors, but old, non-chipped sensors were not compatible with 
        the new monitors. The OxiMax system also let customers swap in new 
        sensor designs without having to replace their monitors, since the 
        critical functions now live in the sensors.
        \item Tyco discontinued the R-Cal system in February 2003. It used two 
        marketing agreements to launch OxiMax:
        \begin{enumerate}
            \item \textbf{Market-share discount agreements}: customers get 
            discounts if they commit to buy a certain percentage of their 
            pulse oximetry equipment from Tyco. The bigger the percentage, the 
            bigger the discount.
            \item \textbf{Sole-source agreements}: group purchasing 
            organizations get a discount if they agree to buy exclusively from 
            Tyco.
        \end{enumerate}
        \item Several companies began selling R-Cal sensors after the patent 
        expired. Tyco's market share dropped from 62\%--64\% to 35\%.
    \end{enumerate}
    \item Plaintiffs (health care providers) argued that the agreements 
    violated \S\S\ 1 and 2 of the Sherman Act.\footnote{995--96.}
    \item The district court held that the agreements did not create an 
    unreasonable restraint on trade (\S\ 1) because the agreements were 
    voluntary and freely terminable. It also held that the introduction of 
    OxiMax and other business tactics did not unreasonably restrict 
    competition (\S\ 2).\footnote{995--96.} OxiMax was just a better product, 
    and Tyco did nothing to force it on its customers.
    \item Marketing agreements and \S\ 1:
    \begin{enumerate}
        \item \S\ 1 outlaws practices that restrain trade or commerce.
        \item The plaintiffs' theory was \textbf{exclusive dealing}, which 
        ``involves an agreement between a vendor and a buyer that prevents the 
        buyer from purchasing a given good from any other 
        vendor.''\footnote{996.}
        \item The agreements did not obligate anybody to buy anything from 
        Tyco. The discount was an incentive, but it didn't force anybody to 
        buy Tyco's products. Vendors of generic sensors could freely compete.
        \item No \S\ 1 violation.\footnote{998.}
    \end{enumerate}
    \item OxiMax and \S\ 2:
    \begin{enumerate}
        \item \S\ 2 outlaws monopolization. Monopolization is (1) the 
        possession of monopoly power in the relevant market, (2) willful 
        acquisition or maintenance of monopoly power, and (3) causal injury.
        \item The relevant market was the oximetry sensor market.
        \item The parties agreed that Tyco was a monopolist. The question was 
        whether it unlawfully maintained its power.\footnote{998.}
        \item \emph{CalComp} and \emph{Foremost}: ``product improvement by 
        itself does not violate Section 2, \textbf{even if it is performed by 
        a monopolist and harms competitors as a 
        result}.''\footnote{999--1000.} Coercive conduct is required.
        \item There is no room for balancing the benefits of the product 
        improvements against its anticompetitive effects. Innovation is 
        \textbf{per se legal} (at least in the Ninth Circuit).
        \item ``~.~.~.~Plaintiffs have provided no evidence that Tyco used its 
        monopoly power to force consumers of pulse oximetry products to adopts 
        its new OxiMax technology. Absent evidence of such compulsion, the 
        only rational inference that can be drawn from some consumers' 
        adoption of OxiMax is that they regarded it to be a superior 
        product.''\footnote{1002.}
        \item No \S\ 2 violation.
    \end{enumerate}
    \item Affirmed.
\end{enumerate}

\subsection{\emph{FTC v. Intel}}

\begin{enumerate}
    \item Intel has 75 to 85\% of the market for PC and server CPUs.
    \item The FTC alleged that to preserve its monopoly, Intel:
    \begin{enumerate}
        \item Punished its customers for using CPUs manufactured by its 
        competitors, AMD and Via.
        \item Used deceptive practices to make it seem that AMD and Via 
        products did not perform as well as they actually did.
    \end{enumerate}
    \item Violations: see p. 17.
\end{enumerate}
