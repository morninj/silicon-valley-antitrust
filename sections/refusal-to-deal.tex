\section{Interconnect Obligations and Refusals to Deal}

\begin{enumerate}
    \item Monopolists generally are not required to deal with anyone, but there 
    are exceptions---e.g., common carrier regulations, discontinuation of a 
    practice (for companies with significant market power---over 
    60\%---\emph{Aspen}).
    \item De novo refusals to deal are fine. Obligations only arise from 
    preexisting profitable courses of dealing.
    \item (The \emph{essential facilities doctrine} lives on in some circuits. 
    But it has no practical significance in the U.S.)
    \item You generally have to show profit sacrifice to prove exclusionary 
    conduct.
\end{enumerate}

\subsection{Irrational Sacrifice---Profit Sacrifice: Discontinuation of a 
Practice: \emph{Aspen Skiing Co. v. Aspen Highlands Skiing Corp.}}

Discontinuation of a practice can count as exclusionary conduct. Here, Ski Co.  
acted irrationally by sacrificing profits. ``If a firm has been attempting to 
exclude rivals on some basis other than efficiency, it is fair to characterize 
its behavior as predatory.''\footnote{605, quoting Bork.}

\emph{Aspen} appears to suggest that irrational conduct (but for an 
anticompetitive motive) is \emph{required} to show exclusionary conduct. But 
\emph{Trinko}, below, called this into question.

\begin{enumerate}
    \item There were four ski mountains in Aspen. Ski Co. operated three of 
    them. Highlands operated the fourth.
    \item For awhile, they agreed to provide an all-Aspen, six-day pass that 
    was valid at all of the resorts. Eventually, Ski Co. refused to continue to 
    offer the ticket with Highlands. Highlands essentially became a day resort.
    \item Highlands alleged that Ski Co.'s refusal to deal was a violation of 
    \S\ 2.
    \item Ski Co. admitted it had monopoly power.\footnote{413.}
    \item Held: Ski Co. \emph{did} have a duty to deal.
    \begin{enumerate}
        \item This was not a de novo refusal to deal. Rather, it was a 
        \textbf{termination of a profitable course of dealing}.\footnote{603.}
        \item Termination is anticompetitive if the monopolist makes a 
        sacrifice. Here, Ski Co. sacrificed profits.\footnote{608.}
    \end{enumerate}
\end{enumerate}

\subsection{No Exclusionary Conduct without Irrationality? \emph{Verizon 
Commun. v. Law Offices of Curtis V. Trinko}}

After \emph{Trinko}, it's unclear whether the plaintiff \emph{must} show that 
the monopolist acted irrationally, or whether irrational conduct \emph{may} 
support a finding of exclusionary conduct.

\begin{enumerate}
    \item The 1996 Telecommunications Act required Verizon, a monopolist, to 
    share its network with competitors.
    \item 1999: Verizon's competitors complained that many service requests to 
    Verizon were going unfilled.
    \item Trinko was a customer of AT\&T, one of Verizon's competitors. He 
    argued that Verizon gave bad service to its competitors in order to 
    encourage their customers to switch to Verizon, in violation of \S\ 2.
    \item Held: Verizon's refusal to deal was not exclusionary because it did 
    not involve a loss of profits.
    \item Unlike Aspen, Verizon (1) was not deviating from a prior course of 
    conduct and (2) did not act irrationally.
\end{enumerate}

novell % TODO 
------

no profit sacrifice (in total--even though msft may have lost some profits in 
the OS market)
    privileges vertically integrated companies
    this rule could justify behavior that is unlawful under other rules -- 
    e.g., how would kodak have argued under this rule? 

