\section{Very Short Overview}

\begin{enumerate}
    \item Monopolies only become problematic when they harm consumers. Scalia: 
    monopolies are good.
\end{enumerate}

\subsection{Sherman Act \S\ 1: Agreements in Restraint of Trade}

\begin{enumerate}
    \item Two prongs: (1) agreement (2) in restraint of trade.
    \item Doesn't just apply to horizontal agreements. Can also apply to 
    tying or exclusive dealing (which are vertical agreements).
    \item \textbf{Per se}: price fixing, market allocation, boycotting. 
    Independent of market power. \emph{Andreas}.
    \begin{enumerate}
        \item Can lead to false positives. If the restrictive agreement is 
        \textbf{ancillary} to an underlying procompetitive transaction, there 
        is no \S\ 1 violation. \emph{BMI v. Columbia}.
        \item Courts will only apply the per se rule where there has been 
        \textbf{significant judicial experience.}
        \item Upstream agreements can be per se unlawful. \emph{U.S. v. 
        Adobe}, \emph{U.S. v. Apple}. % TODO other takeaways from apple?
        \item (Tying can also be a \S\ 1 offense. And it can be a per se 
        offense---but usually not. See \S\ 2 below.)
    \end{enumerate}
    \item \textbf{Rule of reason}: unlawful if anticompetitive effects (AE) 
    are greater than the procompetitive efficiencies (PE).
\end{enumerate}

\subsubsection{Analyzing a Horizontal Agreement Problem}

\begin{enumerate}
    \item Was this a literal \textbf{per se offense} (price fixing, market 
    allocation, boycott)?
    \begin{enumerate}
        \item Yes:
        \begin{enumerate}
            \item Has there been significant judicial experience?
            \begin{enumerate}
                \item Yes: \emph{violation of \S\ 1}.
                \item No: apply rule of reason.
            \end{enumerate}
        \end{enumerate}
        \item No: go to rule of reason analysis.
    \end{enumerate}
    \item Was the agreement \textbf{ancillary} to an underlying procompetitive 
    transaction (as in BMI)?
    \begin{enumerate}
        \item Yes: apply rule of reason.
        \item No: \emph{violation of \S\ 1}.
    \end{enumerate}
    \item \textbf{Rule of reason}:
    \begin{enumerate}
        \item Market power: is the defendant's market share greater than 30\%?
        \item Is there a plausible theory of harm?
        \item Is there proof of harm?
        \item Court asks: AE > PE?
    \end{enumerate}
\end{enumerate}

\subsection{Sherman Act \S\ 2: Abuse of Monopoly Position}

\begin{enumerate}
    \item Requires (1) \textbf{monopoly power} and (2) \textbf{exclusionary 
    conduct}.
\end{enumerate}

\subsubsection{Analyzing a \S\ 2 Monopolization Problem}

Requires monopoly power, exclusionary conduct, and causation.

\begin{enumerate}
    \item Monopoly power.
    \begin{enumerate}
        \item Direct proof, and/or
        \item Indirect proof:
        \begin{enumerate}
            \item Market definition (hypothetical monopolist + small but 
            significant nontransitory increase in price [SSNIP])---i.e., an 
            actor with the ability to profitably raise prices. The market 
            % TODO clarify -- see slides on market def, class 4
            includes reasonable substitutes.
            \begin{enumerate}
                \item The market definition can include \textbf{nascent 
                threats}. \emph{Microsoft}.
            \end{enumerate}
            \begin{enumerate}
                \item Avoid the \emph{Cellophane} fallacy: if a monopolist has 
                had a monopoly for a long time, then it will already be 
                charging a monopoly price, so this test fails. There's no good 
                way to deal with this.
            \end{enumerate}
            \item Significant market share (somewhere around 60\% or more).
            \item Barriers to entry.
            \begin{enumerate}
                \item Network effects can be a barrier to entry. 
                \emph{Microsoft.}
            \end{enumerate}
        \end{enumerate}
    \end{enumerate}
    \item Exclusionary conduct.
    \begin{enumerate}
        \item Does AE > PE?
        % TODO: verify. See HK slides 04-02, slide 10, at http://www.hannokaiser.com/lawschool/2013-fall-svat/slides/2013-svatf-04-02.pdf)
        \item Exclusion of \textbf{interoperability layers} can be 
        exclusionary conduct.
        \item Exclusion of \textbf{nascent threats} can also be exclusionary 
        conduct. Nascent threats may be entitled to stronger protections than 
        small but established competitors. \emph{Microsoft}.
        % maybe predatory innovation
            % intent? probability of acquiring market power?
        \item \textbf{Tying}. \emph{Kodak}, \emph{Datel}. Four elements:
        \begin{enumerate}
            \item Two products.
            \item Tie.
            \item Market power in the \emph{tying} product. \textbf{Be sure to 
            define the right market.}
            \item Some effect in the \emph{tied} product market. Define the 
            market here, too.
            \item (Can be contractual, economic,  technological.)
        \end{enumerate}
        % refusal to deal
            % essential facilities doctrine
        % abuse of standards essential patents
        % google?
    \end{enumerate}
    \item Business justifications.
    \item Causation.
    \begin{enumerate}
        \item The exclusionary conduct creates or reinforces the monopoly 
        power.
    \end{enumerate}
\end{enumerate}

\subsection{Clayton Act \S\ 7: Mergers and Acquisitions}

\begin{enumerate}
    \item No acquisition of shares or assets that would likely result in a 
    ``substantial lessening of competition.''
\end{enumerate}

\subsection{Analyzing a Horizontal Merger Problem}

\begin{enumerate}
    \item % TODO 
\end{enumerate}
