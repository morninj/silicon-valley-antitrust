\section{Very Short Overview}

\begin{enumerate}
    \item Monopolies only become problematic when they harm consumers. Scalia: 
    monopolies are good.
\end{enumerate}

\subsection{Sherman Act \S\ 1: Agreements in Restraint of Trade}

\begin{enumerate}
    \item Two prongs: (1) agreement (2) in restraint of trade.
    \item Doesn't just apply to horizontal agreements. Can also apply to 
    tying or exclusive dealing (which are vertical agreements).
    \item \textbf{Per se}: price fixing, market allocation, boycotting. 
    Independent of market power. \emph{Andreas}.
    \begin{enumerate}
        \item Can lead to false positives. If the restrictive agreement is 
        \textbf{ancillary} to an underlying procompetitive transaction, there 
        is no \S\ 1 violation. \emph{BMI v. Columbia}.
        \item Courts will only apply the per se rule where there has been 
        \textbf{significant judicial experience.}
        \item Upstream agreements can be per se unlawful. \emph{U.S. v. 
        Adobe}, \emph{U.S. v. Apple}. % TODO other takeaways from apple?
        \item (Tying can also be a \S\ 1 offense. And it can be a per se 
        offense---but usually not. See \S\ 2 below.)
    \end{enumerate}
    \item \textbf{Rule of reason}: unlawful if anticompetitive effects (AE) 
    are greater than the procompetitive efficiencies (PE).
\end{enumerate}

\subsubsection{Analyzing a Horizontal Agreement Problem}

\begin{enumerate}
    \item Was this a literal \textbf{per se offense} (price fixing, market 
    allocation, boycott)?
    \begin{enumerate}
        \item Yes:
        \begin{enumerate}
            \item Has there been significant judicial experience?
            \begin{enumerate}
                \item Yes: \emph{violation of \S\ 1}.
                \item No: apply rule of reason.
            \end{enumerate}
        \end{enumerate}
        \item No: go to rule of reason analysis.
    \end{enumerate}
    \item Was the agreement \textbf{ancillary} to an underlying procompetitive 
    transaction (as in BMI)?
    \begin{enumerate}
        \item Yes: apply rule of reason.
        \item No: \emph{violation of \S\ 1}.
    \end{enumerate}
    \item \textbf{Rule of reason}:
    \begin{enumerate}
        \item Market power: is the defendant's market share greater than 30\%?
        \item Is there a plausible theory of harm?
        \item Is there proof of harm?
        \item Court asks: AE > PE?
    \end{enumerate}
\end{enumerate}

\subsection{Sherman Act \S\ 2: Abuse of Monopoly Position}

\begin{enumerate}
    \item Requires (1) \textbf{monopoly power} and (2) \textbf{exclusionary 
    conduct}.
\end{enumerate}

\subsubsection{Analyzing a \S\ 2 Monopolization Problem}

Requires monopoly power, exclusionary conduct, and causation.

\begin{enumerate}
    \item \textbf{Monopoly power}.
    \begin{enumerate}
        \item Direct proof, and/or
        \item Indirect proof:
        \begin{enumerate}
            \item Market definition---the \textbf{SSNIP test}:
            \begin{enumerate}
                \item \textbf{Purpose}: identify the smallest relevant market 
                in which a hypothetical monopolist could profitably impose an 
                increase in price.
                \item \textbf{Test}:
                \begin{itemize}
                    \item Start with the smallest possible market.
                    \item If a hypothetical monopolist controlled all the 
                    products in the market, could it profitably impose a 5\% 
                    price increase?
                    \begin{itemize}
                        \item No (i.e., consumers would substitute other 
                        products): go back to step 1 and expand the market.  
                        \item Yes: this is the relevant market.
                    \end{itemize}
                \end{itemize}
                \item \textbf{Example}:
                \begin{itemize}
                    \item Market \#1: scotch.
                    \item If a hypothetical monopolist raised the price by 5\%, 
                    would consumers switch to something else? Yes—so the 
                    market is too narrow.
                    \item Market \#2: scotch and bourbon.
                    \item If a hypothetical monopolist raised the price by 5\%, 
                    would consumers switch to something else? No—so this is 
                    the relevant market. 
                \end{itemize}
            \end{enumerate}
            \begin{enumerate}
                \item The market definition can include \textbf{nascent 
                threats}. \emph{Microsoft}.
            \end{enumerate}
            \begin{enumerate}
                \item Avoid the \emph{Cellophane} fallacy: if a monopolist has 
                had a monopoly for a long time, then it will already be 
                charging a monopoly price, so this test fails. There's no good 
                way to deal with this.
            \end{enumerate}
            \item Significant market share (somewhere around 60\% or more).
            \item Barriers to entry.
            \begin{enumerate}
                \item Network effects can be a barrier to entry. 
                \emph{Microsoft.}
            \end{enumerate}
        \end{enumerate}
    \end{enumerate}
    \item \textbf{Exclusionary conduct}.
    \begin{enumerate}
        % TODO: verify. See HK slides 04-02, slide 10, at http://www.hannokaiser.com/lawschool/2013-fall-svat/slides/2013-svatf-04-02.pdf)
        \item Exclusion of \textbf{interoperability layers}. \emph{Microsoft}.
        \item Exclusion of \textbf{nascent threats}. Nascent threats may 
        be entitled to stronger protections than small but established 
        competitors. \emph{Microsoft}.
        \item \textbf{Predatory innovation}.
        \begin{enumerate}
            \item \emph{Microsoft} test: plaintiff must make a prima facie 
            case of predation; defendant can rebut; plaintiff must show 
            that the exclusionary effects outweigh the defendant's 
            justifications.
            \item \emph{Allied Ortho} test: any benefit to the consumer 
            establishes \textbf{per se legality}, regardless of predation 
            or effects on the market. Ninth Circuit only.
            \item \emph{FTC v. Intel}: consent decree rejected the 
            \emph{Allied Ortho} test.
        \end{enumerate}
        \item \textbf{Tying}. \emph{Kodak} 
        (p.~\pageref{kodak-tying-claim}), \emph{Datel}. Four elements:
        \begin{enumerate}
            \item \textbf{Two products} (e.g., Kodak copier parts and 
            services).
            \item \textbf{Tie} (e.g., Kodak's policy change to require 
            buyers of Kodak parts to use Kodak services). Can be 
            contractual (\emph{Kodak}), economic, technological.
            \item Market power in the \emph{tying} product (the ``must 
            have'' product---e.g., Kodak parts). \textbf{Be sure to define 
            the right market.}
            \item Some effect in the \emph{tied} product market (e.g., 
            all services for Kodak copiers). Define the market here, too.
        \end{enumerate}
        \item \textbf{Closed systems}: not exclusionary by default, but they 
        can raise antitrust issues, particularly when a system moves from open 
        to closed.
        \item \textbf{Refusal to deal}: monpolists generally are not required 
        to deal with anyone, but there are exceptions---e.g., common carrier 
        regulations, \textbf{discontinuation of a practice} (for companies 
        with significant market power---over 60\%---\emph{Aspen}), and the 
        essential facilities doctrine (which is rarely used). You generally 
        have to show \textbf{profit sacrifice} to show exclusionary conduct. 
        \emph{Trinko}.
        \item \textbf{Abuse of a private standard-setting process} can be 
        actionable. \emph{Broadcom}.
        % TODO google?
    \end{enumerate}
    \item \textbf{Business justifications}. Key question: \textbf{does AE > 
    PE}? (But see \emph{Allied Ortho} just above, under predatory innovation.)
    \item \textbf{Causation}: the exclusionary conduct creates or reinforces 
    the monopoly power.
\end{enumerate}

\subsection{Clayton Act \S\ 7: Mergers and Acquisitions}

\begin{enumerate}
    \item No acquisition of shares or assets that would likely result in a 
    \textbf{``substantial lessening of competition.''} \textbf{Mergers should 
    not be permitted to enhance market power.}
    \item The analysis is generally concerned with \textbf{price effects}, 
    but there can be non-price effects as well---e.g., reduced product 
    quality or increased incentives for exclusionary conduct.
    \item Ex ante notification requirements make \S\ 7 largely an agency 
    practice.
\end{enumerate}

\subsection{Analyzing a Horizontal Merger Problem}

\begin{enumerate}
    \item \textbf{Market definition and market power}.
    \begin{enumerate}
        \item Market definition: see SSNIP test under \S\ 2 above.
        \item How many competitors will there be after the merger? 
        (\emph{Philadelphia National Bank}'s ``structural presumption'': more 
        than 30\% market share creates a presumption of anticompetitive 
        effects).
        \begin{enumerate}
            \item 5 to 4: ok (around 25\% share).
            \item 4 to 3: generally ok (around 33\% share).
            \item 3 to 2: not ok (around 50\% share).
            \item 2 to 1: not ok (monopoly!).
        \end{enumerate}
    \end{enumerate}
    \item Would the buyer gain the \textbf{ability} (e.g., if A and B are 
    competitors, and A acquires their only supplier, S, then A could prevent B 
    from getting the supplies it needs) and \textbf{incentive} (e.g., would A 
    \emph{gain} more from harming B in the downstream market than it would 
    \emph{lose} in sales of supplies to B?) to foreclose competition?
    \begin{enumerate}
        \item \textbf{Unilateral effects}.\footnote{See p. 20 ff.}
        \begin{enumerate}
            \item Elimination of competition between competitors.
            \item \textbf{Pricing of differentiated products}:\footnote{p. 20} 
            if the merging firms sell products that directly compete, the 
            merger could allow them to raise prices on both products. (Four 
            factors---see \emph{Oracle} analysis, 
            p.~\pageref{oracle-unilateral-effects}).
            \item \textbf{Bargaining and auctions}:\footnote{p. 22} if two 
            sellers merger, buyers will no longer be able to play the sellers 
            off of each other.
            \item \textbf{Capacity and output for homogenous 
            products}:\footnote{p. 22} the merged firm may be able to suppress 
            output and raise the market price.
            \item \textbf{Innovation and product variety}: the merged firm may 
            lose incentives to innovate because it won't need to compete on 
            innovation.
        \end{enumerate}
        \item \textbf{Coordinated effects}.\footnote{p. 24 ff.}
        \begin{enumerate}
            \item Increasing the risk of coordinated anticompetitive behavior.
            \item First: is the market moderately or highly 
            concentrated?\footnote{p. 25.}
            \item Second: is the market vulnerable to coordinated 
            conduct?\footnote{p. 25 ff.}
            \item If yes to both, there is a strong risk of coordinated 
            effects.
        \end{enumerate}
    \end{enumerate}
    \item Will the foreclosure result in significant \textbf{harm to the 
    competitive process in the downstream market}? (Unlikely if there are many 
    competitive players.)
    \item Are others \textbf{likely to enter the market}?
    \item Are there any \textbf{procompetitive efficiencies}?
    \begin{enumerate}
        \item \textbf{Economies of scale}.
        \item \textbf{Economies of scope}.
        \item Horizontal mergers are the strongest form of cartel, but they 
        are not per se illegal because of the procompetitive efficiencies.
    \end{enumerate}
\end{enumerate}
